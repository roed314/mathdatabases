\documentclass{article}
\usepackage{amsmath,amssymb}
\usepackage{url}
\usepackage{xcolor}

\newcommand{\todo}[1]{\textcolor{red}{#1}}

\title{Maintaining Mathematical Databases}
\author{Edgar Costa and David Roe}

\begin{document}
\maketitle

\section{Introduction}

Clarify that we're talking about hosting mathematical data, rather than written mathematical content (such as Wikipedia, Mathworld, MathSciNet, arXiv) or something like the Mathematical Genealogy Project

\section{Survey of major mathematical databases}

\begin{itemize}
\item LMFDB (\url{lmfdb.org})
\item OEIS (\url{oeis.org})
\item Encyclopedia of Triangle Centers (\url{http://faculty.evansville.edu/ck6/encyclopedia/ETC.html})
\item Inverse Symbolic Calculator (\url{http://wayback.cecm.sfu.ca/projects/ISC/ISCmain.html})
\item Information System on Graph Classes and their Inclusions (\url{graphclasses.org/})
\item Complexity Zoo (\url{https://complexityzoo.uwaterloo.ca/Complexity_Zoo}) - discussion of source of data: computers or humans
\item Digital Library of Mathematical Functions (\url{https://dlmf.nist.gov/})
\item Consequences of the Axiom of Choice (\url{http://www.math.purdue.edu/~hrubin/JeanRubin/Papers/conseq.html})
\item Cantor's Attice (\url{cantorsattic.info})
\item Unipotent Muffin Research Kitchen (\url{http://lie.math.okstate.edu/UMRK/UMRK.html})
\item Atlas of Lie Groups and Representations (\url{http://www.liegroups.org/})
\item Atlas of Finite Group Representations (\url{http://brauer.maths.qmul.ac.uk/Atlas/v3/})
\item FindStat (\url{http://www.findstat.org/})
\item Matroids (\url{https://www.math.ucdavis.edu/~mkoeppe/art/Matroids/})
\item GraphArichive (\url{http://www.graph-archive.org/doku.php}) - example that died
\item Table of Knot Invariants (\url{http://www.indiana.edu/~knotinfo/})
\item Databases included in GAP/Sage/etc
\item Knot Atlas (\url{http://katlas.org/})
\item Predecessors to the LMFDB (Cremona's tables, \url{https://math.la.asu.edu/~jj/localfields/} and other Jones' databases, \url{https://wstein.org/Tables/})
\item Calabi-Yau data \url{http://hep.itp.tuwien.ac.at/~kreuzer/CY/}
\item The K3 Database \url{https://magma.maths.usyd.edu.au/magma/handbook/text/1410}
\item Fano Varieties and Extremal Laurent Polynomials \url{http://coates.ma.ic.ac.uk/fanosearch/}
\end{itemize}

See question of Tim Gowers (https://mathoverflow.net/questions/68442/what-could-be-some-potentially-useful-mathematical-databases) about what databases might be desirable to have.

\section{Hosting}

\begin{itemize}
\item On a personal webpage (easy for many academics), but limits scope (aren't running your own webserver)
\item Using a web framework (e.g. Github, Cocalc, Wordpress/wikidot/blogs)
\item On a university/personal server.  Complete control, but have to deal with down time, buying new servers, backups, security updates, DNS....
\item Servers in the cloud (e.g. Google, AWS).  Reduces hassle, but costs money.
\end{itemize}

\section{Front End}

\begin{itemize}
\item E-mail me if you want the data
\item HTML front page with downloads (static)
\item Wiki page (dynamic but human generated)
\item within computer algebra system (GAP/Sage/Magma)
\item web application (LMFDB uses Python/Flask/Jinja with javascript/CSS/etc).  Allows for dynamic generation of pages, links, nicer representation of data
\end{itemize}

\section{Back End}

\begin{itemize}
\item static files (plain text, sql-lite, Sage pickles,....)
\item Wiki software
\item document oriented database (e.g. MongoDB: schema free)
\item relational database (e.g. PostgreSQL)
\end{itemize}

\section{MongoDB vs PostgreSQL}

When the LMFDB project began, MongoDBDB was an exciting new option.  As a document-oriented database, it did not require a schema, a boon when creating new collections of mathematical objects where the data requirements only gradually became clear.  Moreover, Python bindings were available, a language that many computational number theorists were familiar with due to its use in SageMath.  It worked well for many years, but as the LMFDB grew various shortcomings of MongoDBbecame more problematic.  Over the last year, we have transitioned from MongoDB to PostgreSQL.

\subsection{Comparing two database systems}

\begin{itemize}
%\item Have to specify a schema in PostgreSQL in advance.  This can be a hassle, but also can prevent errors (e.g. typos in column names, more complicated processing code because data isn't uniform)
%\item MongoDB required more storage space since every document had to include the column names (quantify storage space change; quantify how much we're paying)
%\item Integers limited to 64 bits, so got in the habit of storing many things as strings (which made sorting tricky among other issues)
%\item Array types (not just a jsonb)
%\item MongoDB required more RAM (Wired-Tiger; MMAP). Postgres can be faster with more RAM but works okay with less.
\item Indexes on lists didn't work as we wanted (e.g. list of ramified primes) - would work for "what are all the fields ramified at 5" but not "what are all the fields ramified at 2 and 3".  Also inefficient (constructs query inefficiently since there are many more fields than primes).  Postgres has multiple index types, including gin which work well for this purpose
\item Performance improvements, important since we don't know what kind of queries the user will request so it's hard to build all indexes.  Partly because of lower storage, performance also improved even without indexes.  Non-typical queries can be the most interesting mathematically.
\item Harder to do relational queries in MongoDB; since we ported the table layout relational queries are still not used much in Postgres
\item Postgres overcame much of MongoDB's advantage with the inclusion of the json and jsonb types.
\item MongoDB was faster at counting results.  Work around this by caching counts in a table (possible since data doesn't change frequently)
\item Useful new utilities (pgAdmin 4)
\item Fast loading from a file (COPY FROM), and direct data from Magma without a Python intermediary
  add estimate
\item (TODO) Actually compare performance while we still have the mongo instance around.
\item Transactions (bugs don't cause data reliability problems) and data upload much faster
\end{itemize}

PostgreSQL is a popular open source implementation of the SQL language and grew out of earlier projects founded in the 1970s and 80s.
As an SQL database, it stores data in tables with a specified schema: each row must have the same layout, with the types of columns constant across all rows.
In contrast, the fields present in a MongoDB document can vary across a collection, as can their types.
This flexibility is convenient, but can easily lead to errors.
Multiple typos in column names were found during the transition to PostgreSQL, and differing layouts across documents requires more complicated processing code.
Moreover, the LMFDB was effectively using a schema even before the transition, to the extent that an inventory application was written to extract the schema from the data in MongoDB.
Supporting the schema at the level of the database software improves robustness and performance.

As a consequence of the shift to a schema, storing the data in PostgreSQL requires substantially less space than in MongoDB, as we do not need to store the keys for each row.
\todo{Quantify: how much storage space is used in each system?}
For example, in table with 78 columns the saving  \todo{finish sentence}
Two main factors account for this improvement.
Because the fields in a MongoDB document can vary, the names of these fields must be stored in each document.
For some of the LMFDB's collections, these names accounted for a substantial fraction of the storage requirement.
In addition to the space benefits, the switch to PostgreSQL also relaxes the pressure to minimize the lengths of these names, improving readability.

The second factor contributing to space savings is the reduced use of strings in the data itself.
As a mathematical database, the LMFDB was heavily impacted by MongoDB's lack of support for arbitrary precision integers.
In order to work around this issue, a lot of numerical data was stored as strings.
In addition to the storage consequences, various workarounds were required in order to sort data correctly.
The elimination of these hacks have simplified a lot of supporting code.
\todo{Check to see if MongoDB is less efficient than PostgreSQL when storing lists of integers; are they using a string-based json format or a compact array format?}

The two systems also have different requirements in terms of working memory.
\todo{Describe the MongoDB memory requirements}
In contrast, Postgres will work in low memory environments.
\todo{Describe the minimum memory required to run the LMFDB, as well as how additional memory impacts performance}

One of the main tools for improving search performance on databases is the use of indexes.
Indexes facilitate the location of records with constraints on a specified set of columns by storing additional data.
For example, both MongoDB and PostgreSQL support indexes based on binary trees, which work well for totally ordered data such as integers and strings.
However, there is another query type that is not well supported by binary trees.
If $K$ is a number field, the LMFDB stores a list $P_K$ of ramified primes.
We would like to be able to specify a list $Q$ of primes and be able to find all $K$ with $Q \subseteq P_K$, or all $K$ with $Q \supseteq P_K$.
MongoDB executed these searches by narrowing the results one prime at a time \todo{confirm that this is actually what MongoDB is doing}
In contrast, GIN indexes in PostgreSQL support such queries efficiently.

\subsection{Transition}

\begin{itemize}
\item Exporting data to plain text, readable using PostgreSQL' COPY FROM.
\item Choice of schema.  LMFDB's Inventory was very useful here.  Had to choose between jsonb and arrays; initially chose to just use jsonb for simplicity but more recently have used arrays for smaller storage footprint.  Tried to normalize data to some extent (away from using strings for so much). 
\item Data irregularities slowed the process down
\item Added abstraction layers.  psycopg2 is very low level (provides a way to execute SQL commands from Python and get the results back).  Because the lmfdb has almost no queries that cross multiple databases, was able to port MongoDB dictionary queries: implemented a parser that generates simple SQL.  Means that most LMFDB developers don't need to learn SQL.
\item Build data management tools into the interface (changing schema, importing new data from Python and from files).  Replaces parts of scripts.
\item Some optimizations (id ordering, split some tables in half)
\end{itemize}

\section{Benefits}

\subsection{Abstraction}

\begin{itemize}
\item Copy-pasta
\item Statistics
\item Search-parsing
\item Search-wrapper (easier to display helpful error messages if the search times out or has a problem processing arguments)
\end{itemize}

\subsection{New Features}

In addition to intrinsic benefits of using PostgreSQL, the transition has eased the implementation of a number of new features
\begin{itemize}
\item Transactions for database actions
\item Dynamic statistics
\item Random result
\item Log database changes for accountability (possible because of abstraction)
\item Automatic encoding and decoding of more complicated data structures such as number fields
\item Verifiability
\end{itemize}

\end{document}
